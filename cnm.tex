\section{Crank-Nicolson method}
We solve this partial differential equation numerically by using the Crank-Nicolson method (CNM). CNM is a finite difference method that is implicit in time and numerically stable \cite{cnm}. 

The time derivative on the left hand side of equation~\ref{eq:schrodinger} is approximated with a forward difference divided by the time step $\tau$:
\begin{gather*}
    \frac{\partial}{\partial t}\Psi 
    = \frac{\Delta_h[\Psi](t)}{\tau} + \mathcal{O}(\tau)
    = \frac{\Psi(t + \tau) - \Psi(t)}{\tau} + \mathcal{O}(\tau).
\end{gather*} Rewriting equation~1 then gives:
\[
    \Psi(t+\tau)-\Psi(t) 
    = -i\tau\hat{H}\Psi(t).
\] By approximating $\Psi(t)$ as $\left(\Psi(t+\tau)+\Psi(t)\right)/2$ this equation can be rewritten as:
\begin{equation}\label{eq:cnm_lineq}
    \left(\mathbb{1}+\frac{i\tau}{2}\hat{H}\right)\Psi(t+\tau) 
    = \left(\mathbb{1}-\frac{i\tau}{2}\hat{H}\right)\Psi(t).
\end{equation} This is a system of linear equations of the form $\mathbf{A}\vec{x}=\vec{b}$. The wave function can be evaluated by solving this for $\vec{x}$. In these simulations we solve for $\Psi(t+\tau)$ by using the Biconjugate gradient method \footnote{We use the bicgstab function from the Python module Scipy}.

When equation \ref{eq:cnm_lineq} is rewritten as:
\begin{gather*}
\Psi(t+\tau) 
    = \frac{\mathbb{1}-\frac{i\tau}{2}\hat{H}}{\mathbb{1}+\frac{i\tau}{2}\hat{H}}\Psi(t)
\end{gather*} we note the close correspondence (up to second order in $\tau$) between the series expansion of this operator and the series expansion of the time evolution operator (when $\hat{H}(t) = \hat{H}$:

%  If the Hamiltonian is independent of time, the operator acting on $\Psi(t)$ needs to be calculated only once.
%  Where the term in the denominator implies that the inverse of $\mathbb{1}+\frac{i\tau}{2}\hat{H}$ needs to be calculated everytime step if $\hat{H}$ depends on time.

\begin{gather*}
e^{-i\tau\hat{P}} = 1 - i \tau\hat{P} - \frac{\tau^2}{2}\hat{P}^2 + \frac{i\tau^3}{6}\hat{P}^3 + \mathcal{O}(\tau^4)\\
\frac{\mathbb{1}-\frac{i\tau}{2}\hat{H}}{\mathbb{1}+\frac{i\tau}{2}\hat{H}} = 1 - i \tau\hat{P} - \frac{\tau^2}{2}\hat{P}^2 + \frac{i\tau^3}{4}\hat{P}^3 + \mathcal{O}(\tau^4).\\
\end{gather*}

Since

\begin{gather*}
\left(\frac{\mathbb{1}-\frac{i\tau}{2}\hat{H}}{\mathbb{1}+\frac{i\tau}{2}\hat{H}}\right)^{\dagger}\frac{\mathbb{1}-\frac{i\tau}{2}\hat{H}}{\mathbb{1}+\frac{i\tau}{2}\hat{H}}\\
= \frac{\mathbb{1}+\frac{i\tau}{2}\hat{H}}{\mathbb{1}-\frac{i\tau}{2}\hat{H}}\frac{\mathbb{1}-\frac{i\tau}{2}\hat{H}}{\mathbb{1}+\frac{i\tau}{2}\hat{H}} = \mathbb{1},
\end{gather*}

the operator is unitary. This implies that the norm is preserved, neglecting finite computational precision.


\begin{gather*}
\Psi(t+\tau) = e^{-i\tau\hat{H}}\Psi(t)\\
e^{-i\tau\hat{H}} = \mathbb{1}-\frac{i}{2}\tau\hat{H}+\mathcal{O}(t^2)
\end{gather*}

We verified the simulation results by comparing them with analytical solutions for specific potentials. Well-known examples of potentials with analytical solutions are the infinite square well, barrier potential and harmonic oscillator potential.
The infinite square well finds it physical realisation in studying the behaviour of electrons confined in a metal. The barrier potential can be used as a model for understanding quantum tunneling effects. The harmonic oscillator is an ubiquitous form for the potential, finding its use mostly in approximating complex potentials near their local minima.
In two dimensions a double slit potential is used to simulate the purely quantum-mechanical effect of interference.
