\section{Crank-Nicolson method}
We solve this partial differential equation numerically by using the Crank-Nicolson method (CNM). CNM is a finite difference method that is implicit in time and numerically stable \cite{cnm}. 

The time derivative on the left hand side of equation~\ref{eq:schrodinger} is approximated with a forward difference divided by the time step $\tau$:
\begin{gather*}
    \frac{\partial}{\partial t}\Psi 
    = \frac{\Delta_h[\Psi](t)}{\tau} + \mathcal{O}(\tau)
    = \frac{\Psi(t + \tau) - \Psi(t)}{\tau} + \mathcal{O}(\tau).
\end{gather*} Rewriting equation~1 then gives:
\[
    \Psi(t+\tau)-\Psi(t) 
    = -i\tau\hat{H}\Psi(t).
\] By approximating $\Psi(t)$ as $\left(\Psi(t+\tau)+\Psi(t)\right)/2$ this equation can be rewritten as:
\begin{equation}\label{eq:cnm_lineq}
    \left(\mathbb{1}+\frac{i\tau}{2}\hat{H}\right)\Psi(t+\tau) 
    = \left(\mathbb{1}-\frac{i\tau}{2}\hat{H}\right)\Psi(t).
\end{equation} This is a system of linear equations of the form $\mathbf{A}\vb{x}=\vb{b}$. The wave function can be evaluated by solving this for $\vb{x}$. In these simulations we solve for $\Psi(t+\tau)$ by using the Biconjugate gradient method \footnote{We use the bicgstab function from the Python module Scipy.}.

When equation \ref{eq:cnm_lineq} is rewritten as:
\begin{gather*}
\Psi(t+\tau) 
    = \frac{\mathbb{1}-\frac{i\tau}{2}\hat{H}}{\mathbb{1}+\frac{i\tau}{2}\hat{H}}\Psi(t)
\end{gather*} we note the close correspondence between the series expansion of this operator and the series expansion of the time evolution operator (when $\hat{H}$ is time-independent and up to second order in $\tau$):
\begin{gather*}
    e^{-i\tau\hat{H}} 
    = 1 - i \tau\hat{H} - \frac{\tau^2}{2}\hat{H}^2 + \frac{i\tau^3}{6}\hat{H}^3 + \mathcal{O}(\tau^4)\\
    \frac{\mathbb{1}-\frac{i\tau}{2}\hat{H}}{\mathbb{1}+\frac{i\tau}{2}\hat{H}} 
    = 1 - i \tau\hat{H} - \frac{\tau^2}{2}\hat{H}^2 + \frac{i\tau^3}{4}\hat{H}^3 + \mathcal{O}(\tau^4).\\
\end{gather*}
%  If the Hamiltonian is independent of time, the operator acting on $\Psi(t)$ needs to be calculated only once.
%  Where the term in the denominator implies that the inverse of $\mathbb{1}+\frac{i\tau}{2}\hat{H}$ needs to be calculated everytime step if $\hat{H}$ depends on time.
When setting $t=0$ and $\tau=t$, this allows us to write:
\[
    \Psi(t)
    = e^{-it \hat{H}} \Psi(0).
\] The inner product of $Psi(t)$ then becomes
\[
    \braket{\Psi(t)}{\Psi(t)}
    = \ev*{\Psi(0)}{e^{it\hat{H}} e^{-it\hat{H}}}
    = 1,
\] where the operator is unitary. This implies that the norm is preserved, neglecting finite computational precision.

% \begin{gather*}
% \left(\frac{\mathbb{1}-\frac{i\tau}{2}\hat{H}}{\mathbb{1}+\frac{i\tau}{2}\hat{H}}\right)^{\dagger}\frac{\mathbb{1}-\frac{i\tau}{2}\hat{H}}{\mathbb{1}+\frac{i\tau}{2}\hat{H}}\\
% = \frac{\mathbb{1}+\frac{i\tau}{2}\hat{H}}{\mathbb{1}-\frac{i\tau}{2}\hat{H}}\frac{\mathbb{1}-\frac{i\tau}{2}\hat{H}}{\mathbb{1}+\frac{i\tau}{2}\hat{H}} = \mathbb{1},
% \end{gather*}
% \begin{gather*}
% \Psi(t+\tau) = e^{-i\tau\hat{H}}\Psi(t)\\
% e^{-i\tau\hat{H}} = \mathbb{1}-\frac{i}{2}\tau\hat{H}+\mathcal{O}(t^2)
% \end{gather*}

\section{Hamiltonian operator}