\section{Introduction}
The (non-relativistic) Schr\"{o}dinger equation for the wavefunction $\Psi(x,t)$ of a particle in a time-dependent scalar potential $V(x,t)$ reads 

\[
i\hbar \frac{\partial}{\partial t}\Psi(x,t) = \left[-\frac{\hbar^2}{2m}\nabla^2+\hat{V(x,t)}\right]\Psi(x,t),
\]

where $\hbar$ is the quantum of action divided by $2\pi$, $m$ the mass of the particle and $\Psi(x,t)$ the wavefunction.
Rescaling the units such that $\hbar = 2m = 1$ gives

\begin{gather*}
i \frac{\partial}{\partial t}\Psi = \hat{H}(x,t)\Psi\\
\hat{H} = -\nabla^2+\hat{V},
\end{gather*}

where $\hat{H}$ is called the Hamiltonian of the system.

Numerical results can be verified by comparing them with analytical solutions for specific potentials. Well-known examples of potentials with analytical solutions are the infinite square well, barrier potential and harmonic oscillator potential.

The infinite square well finds it physical realisation in studying the behaviour of electrons confined in a metal. The barrier potential can be used as a model for understanding quantum tunneling effects. The harmonic oscillator is an ubiquitous form for the potential, finding its use mostly in approximating complex potentials near their local minima.

