\section{Implicit Crank-Nicolson method}
\begin{gather*}
i\frac{\partial}{\partial t}\Psi = \hat{H}(x,t)\Psi\\
\frac{\partial}{\partial t}\Psi = \lim_{\tau\rightarrow 0}\frac{\Psi(x,t+\tau)-\Psi(x,t)}{\tau} \approx \frac{\Psi(x,t+\tau)-\Psi(x,t)}{\tau}+\mathcal{O}(\tau)\\
i\frac{\Psi(t+\tau)-\Psi(t)}{\tau} = \hat{H}\Psi(t)\\
\Psi(t+\tau)-\Psi(t) = -i\tau\hat{H}\Psi(t)\\
\end{gather*}

The Crank-Nicolson method approximates $\hat{H}\Psi(t)$ as  $\hat{H}\frac{\Psi(t+\tau)+\Psi(t)}{2}$, so that

\begin{gather*}
\Psi(t+\tau)-\Psi(t) = -i\tau\hat{H}\frac{\Psi(t+\tau)+\Psi(t)}{2}\\
\left(\hat{\mathrm{I}}+\frac{i\tau}{2}\hat{H}\right)\Psi(t+\tau) = \left(\hat{\mathrm{I}}-\frac{i\tau}{2}\hat{H}\right)\Psi(t)\\
\Psi(t+\tau) = \frac{\hat{\mathrm{I}}-\frac{i\tau}{2}\hat{H}}{\hat{\mathrm{I}}+\frac{i\tau}{2}\hat{H}}\Psi(t)\\
\end{gather*}

Where the term in the denominator implies that the inverse of $\hat{\mathrm{I}}+\frac{i\tau}{2}\hat{H}$ needs to be calculated everytime step if $\hat{H}$ depends on time. If the Hamiltonian is independent of time, the operator acting on $\Psi(t)$ needs to be calculated only once. Note the close correspondence (up to second order in $\tau$) between the series expansion of this operator and the series expansion of the time evolution operator (when $\hat{H}(t) = \hat{H}$:

\begin{gather*}
e^{-i\tau\hat{P}} = 1 - i \tau\hat{P} - \frac{\tau^2}{2}\hat{P}^2 + \frac{i\tau^3}{6}\hat{P}^3 + \mathcal{O}(\tau^4)\\
\frac{\hat{\mathrm{I}}-\frac{i\tau}{2}\hat{H}}{\hat{\mathrm{I}}+\frac{i\tau}{2}\hat{H}} = 1 - i \tau\hat{P} - \frac{\tau^2}{2}\hat{P}^2 + \frac{i\tau^3}{4}\hat{P}^3 + \mathcal{O}(\tau^4).\\
\end{gather*}

Since

\begin{gather*}
\left(\frac{\hat{\mathrm{I}}-\frac{i\tau}{2}\hat{H}}{\hat{\mathrm{I}}+\frac{i\tau}{2}\hat{H}}\right)^{\dagger}\frac{\hat{\mathrm{I}}-\frac{i\tau}{2}\hat{H}}{\hat{\mathrm{I}}+\frac{i\tau}{2}\hat{H}}\\
= \frac{\hat{\mathrm{I}}+\frac{i\tau}{2}\hat{H}}{\hat{\mathrm{I}}-\frac{i\tau}{2}\hat{H}}\frac{\hat{\mathrm{I}}-\frac{i\tau}{2}\hat{H}}{\hat{\mathrm{I}}+\frac{i\tau}{2}\hat{H}} = \hat{\mathrm{I}},
\end{gather*}

the operator is unitary. This implies that the norm is preserved, neglecting finite computational precision.


\begin{gather*}
\Psi(t+\tau) = e^{-i\tau\hat{H}}\Psi(t)\\
e^{-i\tau\hat{H}} = \hat{\mathrm{I}}-\frac{i}{2}\tau\hat{H}+\mathcal{O}(t^2)
\end{gather*}